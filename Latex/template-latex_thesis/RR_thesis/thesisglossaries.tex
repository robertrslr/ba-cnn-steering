% !TEX root = ./thesis.tex
% List of all symbols, acronyms and glossarie entries
% @author Thomas Lehmann
%

% Warning: Lable must only be used once

% Glossarie entries
\newglossaryentry{gl:haw}{
  name={HAW Hamburg}, 
  description={Die HAW Hamburg ist die formalige Fachhochschule am Berliner Tor}}
  
\newglossaryentry{gl:nuc}{
  name={Intel NUC}, 
  description={Kleiner Barebone-Computer entwickelt von der Firma Intel}}
  
\newglossaryentry{gl:cnn}{
  name={CNN}, 
  description={Convolutional Neural Network, ein neuronales Netz mit Faltungsoperationen}}
  
\newglossaryentry{gl:ids}{
  name={IDS}, 
  description={Hersteller von Industriekameras}}
  
\newglossaryentry{gl:opencv}{
  name={OpenCV}, 
  description={OpenCV ist eine Bibliothek, die Funktionen für Bilderverarbeitung und Maschinelles Sehen anbietet}}
  
\newglossaryentry{gl:anaconda}{
  name={Anaconda}, 
  description={Open Source Python Distribution für wissenschaftliches Programmieren}}
  
\newglossaryentry{gl:spyder}{
  name={Spyder}, 
  description={Open-Source IDE für wissenschaftliches Programmieren mit Python}}
  
\newglossaryentry{gl:vscode}{
  name={VSCode}, 
  description={Code Editor für Windows, Linux und macOS}}
  
\newglossaryentry{gl:pycharm}{
  name={PyCharm}, 
  description={IDE speziell für Python}}
  
\newglossaryentry{gl:uds}{
  name={UDS}, 
  description={Socket Kommunikation mit File-Deskriptoren}}
  
  \newglossaryentry{gl:alvinn}{
  name={ALVINN}, 
  description={Bezeichnung für ein neuronales Netz der Carnegie Mellon Universität}}
  
   \newglossaryentry{gl:navlab}{
  name={NAVLAB}, 
  description={Eine Serie autonomer und semi-autonomer Fahrzeuge der Robotik-Gruppe der Carnegie Mellon Universität}}
  
     \newglossaryentry{gl:eth}{
  name={ETH Zürich}, 
  description={Eidgenössische Technische Hochschule Zürich}}
  
     \newglossaryentry{gl:carolo}{
  name={Carolo-Cup}, 
  description={Studentischer Wettbewerb der TU Braunschweig, Modellfahrzeuge treten in Disziplinen autonom gegeneinander an}}
  
   \newglossaryentry{gl:udacity}{
  name={Udacity}, 
  description={Private Online-Bildungsorganisation gegründet von Sebastian Thrun}}
  
  \newglossaryentry{gl:dronet}{
  name={DroNet}, 
  description={Bezeichnung für ein neuronales Netz, entwickelt von einer Arbeitsgruppe der ETH Zürich}}
  
    \newglossaryentry{gl:adam}{
  name={Adam}, 
  description={Ein Optimierungsalgorithmus für stochastische Gradientenverfahren}}
  
    \newglossaryentry{gl:hardnegative}{
  name={Hard Negative Mining}, 
  description={Ein Optimierungsalgorithmus für stochastische Gradientenverfahren}}




% Acronyms
\newacronym{ac:haw}{HAW}{Hochschule für Angewandte Wissenschaften}
\newacronym[see={[Glossary:]{gl:nuc}}]{ac:nuc}{NUC}{Next Unit of Computing\glsadd{gl:nuc}}
\newacronym[see={[Glossary:]{gl:ids}}]{ac:ids}{IDS}{Imaging Development Systems GmbH\glsadd{gl:ids}}
\newacronym[see={[Glossary:]{gl:uds}}]{ac:uds}{UDS}{Unix Domain Sockets\glsadd{gl:uds}}
\newacronym[see={[Glossary:]{gl:alvinn}}]{ac:alvinn}{ALVINN}{Autonomous Land Vehicle In a Neural Network\glsadd{gl:alvinn}}
\newacronym[see={[Glossary:]{gl:vscode}}]{ac:vscode}{VSCode}{Virtual Studio Code\glsadd{gl:vscode}}
\newacronym[see={[Glossary:]{gl:cnn}}]{ac:cnn}{CNN}{Convolutional Neural Network\glsadd{gl:cnn}}




% Symbols
\newglossaryentry{sy:ohm}{ 
  type=symbols, 
  name={\ensuremath{\Omega}}, 
  sort=Ohm, symbol={\ensuremath{\Omega}}, 
  description={unit of electrical resistance}}
