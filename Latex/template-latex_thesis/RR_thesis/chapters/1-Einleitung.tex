% first example chapter
% @author Jan Robert Rösler 
%


\chapter{Einleitung}



Sobald ein System, welcher Art sei offen, mobil wird, also läuft, rollt, gleitet, schwebt oder schwimmt, steht es vor der Aufgabe der Navigation. Das kann zunächst bedeuten, zu Wissen, wo es sich befindet. Auf einer Karte oder auch relativ zu anderen Objekten in der Umgebung. In der Robotik nennt man diese Kompetenz Lokalisation. Wenn man weiß wo man ist, könnte sich zusätzlich die Frage stellen, wie man zu einem bestimmten Ort hinkommt. Je nach Ziel oder Aufgabe des mobilen Systems, ist die in der Robotik Pfadplanung genannte Kompetenz von Bedeutung.\\ Um sich in einer Umgebung zu bewegen, kann es ebenfalls interessant sein, eine eigene Repräsentation der Umgebung zu erstellen. Das Aufbauen einer Karte ist in der Robotik das Mapping und beinhaltet als Disziplin auch das Interpretieren und Auswerten von den gesammelten Umgebungsinformationen in einer Karte.\\
Es wird vorausgesetzt. dass ein solches System bzw. ein solcher Roboter keinerlei Navigationshilfe (z.B. Steuersignale) von außen erfährt, sondern völlig autonom Entscheiden und navigieren muss.\\
Als Grundstein der Entwicklung solcher autonomer Systeme kann man die Erfindung von dem Neurophysiologen William Grey Walter festmachen \cite{walter1950imitation}. Seine in den späten 40er Jahren entwickelten und \glqq Elmer \grqq{} und \glqq Elsie\grqq{} getauften Roboter, sollten ihm dabei helfen, das menschliche Gehirn besser zu verstehen. Beide Roboter, wegen ihrer Form auch Schildkröten genannt, konnten mit je einem Licht- und Berührungssensor, die wiederum jeweils mit einem Motor verbunden waren, unter anderem um Hindernisse herum navigieren. Die getrennte Ansteuerung der Motoren sollte Neuronen im Gehirn simulieren. Diese Gehirnanalogie findet sich in moderner Weise in dieser Arbeit wieder, Neuronale Netze werden in dieser Arbeit eine zentrale Rolle spielen.\\
Navigation in der Robotik setzt sich also aus verschiedenen Unterdisziplinen zusammen, die in ihrem Zusammenspiel ganz besonders aktuell viel Beachtung finden: Fahrzeuge der Firma Tesla sind bereits auf öffentlichen Straßen autonom unterwegs. Teil der technischen Austattung der Fahrzeuge (neben Ultrschallsensoren und Radar) sind Kameras zur Erfassung der Umgebung. \\
In dieser Arbeit wird gerade die visuelle, bildbasierte Navigation die zweite zentrale Rolle haben.


\section{Autonome Navigation mit Bilddaten}
\label{sec:Autonome Navigation mit Bilddaten}
Legt man der Navigationsaufgabe als einzigen Lösungsraum die Bilddaten einer Kamera zugrunde, dann bieten sich verschiedene Möglichkeiten an, diese zu verwenden. Objekterkennung ist häufig der erste Schritt, um sich in einer Umgebung zu orientieren und herauszufinden, aus was sich diese Umgebung überhaupt zusammensetzt. Zusätzlich muss entschieden werden, welche Teile dieser aus Objekten und Flächen aufgebauten Umwelt überhaupt befahrbar ist.
Für Systeme, die in Szenarien verwendet werden sollen, in denen auf Fahrbahnen, Straßen oder andersartig begrenzten Strecken navigiert werden soll, ist die Erkennung dieser Strecke die Schlüsselfunktion. In diesen Fällen bestimmt die Umgebung direkt, wo überhaupt gefahren werden darf, für die Navigation eine enorme Hilfe.\\
Um genau solche Szenarien soll es in dieser Arbeit gehen, speziell um das Fahren auf einer markierten Fahrbahn, die durch eine Kamera am Fahrzeugsystem erfasst wird.




