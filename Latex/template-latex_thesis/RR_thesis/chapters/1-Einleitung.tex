% first example chapter
% @author Jan Robert Rösler 
%


\chapter{Einleitung}


Sobald ein System, welcher Art sei offen, mobil wird, also läuft, rollt, gleitet, schwebt oder schwimmt, steht es vor der Aufgabe der Navigation. Das kann zunächst bedeuten, zu Wissen, wo es sich befindet. Auf einer Karte oder auch relativ zu anderen \glqq Dingen\grqq{}  in der Umgebung. Weiter können sich dann Fragen der Pfadplanung stellen, je nach Ziel oder Aufgabe des mobilen Systems. Ebenfalls könnte es dann von Interesse zu sein, eine eigene Repräsentation (oder Interpretation) der Umgebung aufzubauen und zu speichern, um Lokalisation und Pfadplanung kontinuierlich zu betreiben. Je nach Art und Aufgabe des Systems sind konkrete Probleme in der Navigation und ihre Lösungen zum Beispiel \note{(NAVIGATIONSPROBLEME UND LOESUNG KLASSISCH)} //
Es wird vorausgesetzt, dass das System keinerlei Eingriff von außen erlaubt, es völlig alleine Probleme lösen und Entscheidungen treffen muss, also autonom agiert.



\section{Autonome Navigation}

Nicht erst seit Elon Musk mit seiner Firma Tesla das autonome Fahren zu einem aktuellen und gesellschaftlichen Thema gemacht hat, ist das Thema in der Forschung von Bedeutung.

\note{Überleitung von }


