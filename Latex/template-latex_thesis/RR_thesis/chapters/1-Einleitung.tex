% first example chapter
% @author Jan Robert Rösler 
%


\chapter{Einleitung}



Sobald ein System, welcher Art sei offen, mobil wird, also läuft, rollt, gleitet, schwebt oder schwimmt, steht es vor der Aufgabe der Navigation. Das kann zunächst bedeuten, zu Wissen, wo es sich befindet. Auf einer Karte oder auch relativ zu anderen Objekten in der Umgebung. In der Robotik nennt man diese Kompetenz Lokalisation. Wenn man weiß wo man ist, könnte sich zusätzlich die Frage stellen, wie man zu einem bestimmten Ort hinkommt. Je nach Ziel oder Aufgabe des mobilen Systems, ist die in der Robotik Pfadplanung genannte Kompetenz von Bedeutung.\\ Um sich in einer Umgebung zu bewegen, kann es ebenfalls interessant sein, eine eigene Repräsentation der Umgebung zu erstellen. Das Aufbauen einer Karte ist in der Robotik das Mapping und beinhaltet als Disziplin auch das Interpretieren und Auswerten von den gesammelten Umgebungsinformationen in einer Karte.\\
Es wird vorausgesetzt. dass ein solches System bzw. ein solcher Roboter keinerlei Navigationshilfe (z.B. Steuersignale) von außen erfährt, sondern völlig autonom Entscheiden und navigieren muss.\\
Als Grundstein der Entwicklung solcher autonomer Systeme kann man die Erfindung von dem Neurophysiologen William Grey Walter festmachen. Seine so getauften \glqq Schildkröten \grqq{} sollten ihm dabei helfen
\note{erste navigationen} von hier kommt die gehjirn verbindung(brain cells)		walter1950imitation
Navigation in der Robotik setzt sich also aus verschiedenen Unterdisziplinen zusammen, die in ihrem Zusammenspiel ganz besonders aktuell viel Beachtung finden: Fahrzeuge der Firma Tesla sind bereits auf öffentlichen Straßen autonom unterwegs. Teil der technischen Austattung der Fahrzeuge (neben Ultrschallsensoren und Radar) sind Kameras zur Erfassung der Umgebung. \\
In dieser Arbeit wird gerade die visuelle, bildbasierte Navigation die zweite Zentrale Rolle haben.


\section{Autonome Navigation mit Bilddaten}
\label{sec:Autonome Navigation mit Bilddaten}




