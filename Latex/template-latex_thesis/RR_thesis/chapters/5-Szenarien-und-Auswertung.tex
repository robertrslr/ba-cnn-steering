% first example chapter
% @author Jan Robert Rösler 
%
\chapter{Szenarien und Auswertung}

\subsection{Metriken}


Auch Geschwindigkeiten beachten, Carolo-Net performt bis 1.2 m/s
\subsection{Testfahrten}

1. Auto mit Dronet 
2. Auto mit adaptiertem Netz
3. Auto mit Netz des aktuellen Carolo-Cup Teams 

\subsection{Sonderszenarien}

\paragraph{Kreuzung}

\paragraph{Fahrbahn wiederfinden}




- Findet das Auto auf die Strecke zurück? Schräg ansetzen und autonomen Modus starten
(Zeigt ob Strecke nur auswensig gelernt oder nicht, da  diese blickwinkel nicht im Trainingsset enthalten sind)

\subsection{Saliency}
- Saliency Besipiele Zeigen was für Features erlernt wurden. (Eventuell auch von vorangegangenen Layern)

-Vielleicht noch ein Szenario in dem das Validation set exra schwer (Artefakte) gemacht wurde

\note{Auf die Problematik mit Testsetgröße eingehen, kleines set daher nur letzter blockj trainiert etc....}