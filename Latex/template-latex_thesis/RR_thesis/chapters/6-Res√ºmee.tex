% first example chapter
% @author Jan Robert Rösler 
%
\chapter{Resümee}

Das Ziel dieser Arbeit war, eine bildbasierte neuronale Steuerung für ein RC-Fahrzeug zu entwickeln, was gelungen ist. Durch Anpassung (Fine Tuning) eines bereits trainierten Convolutional Neural Network entstand eine zunächst für das Testszenario geeignete Steuerung. Fahrten auf einer größeren Strecke konnten mangels Verfügbarkeit noch nicht durchgeführt werden. Zukünftige Testfahrten könnten unter Umständen auf der Teststrecke der TU Braunschweig durchgeführt werden, die ganzjährig aufgebaut ist.\\
Im Performance-Vergleich auf Testfahrten schneidet das CNN hervorragend ab. Die Teststrecke kann nicht nur fast fehlerfrei durchfahren werden, im direkten Vergleich mit dem Steuerungsalgorithmus des Carolo-Cup Teams des WS18/19 liegt die hier entwickelte neuronale Steuerung vorn. Ein zusätzlicher Vergleich mit dem Ursprungsnetz DroNet zeigt, das das Fine Tuning die Lenkwinkelbestimmung entscheidend verbessert hat. Einschränkend muss hier angemerkt werden, dass das Fine Tuning eine Anpassung auf einen eingeschränkten, störungsarmen Testbereich gemacht hat.\\
Zusätzlich 

Darüberhinaus






- Teststrecke nicht lang genug 
-Teststrecke zu sauber, d.h. zu optimal zum lernen (Streckenführung perfekt und sauber)'

- Trainingsdaten sammeln optimieren

- 