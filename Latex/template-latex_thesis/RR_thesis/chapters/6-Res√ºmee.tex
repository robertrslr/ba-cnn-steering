% first example chapter
% @author Jan Robert Rösler 
%
\chapter{Resümee}

Das Ziel dieser Arbeit war, eine bildbasierte neuronale Steuerung für ein RC-Fahrzeug zu entwickeln, was gelungen ist. Durch Fine-Tuning eines bereits trainierten Convolutional Neural Network (DroNet) entstand eine geeignete Steuerung. Zur Beurteilung der Qualität der Steuerung wurden Fahrten auf einer Teststrecke durchgeführt. Fahrten auf einer größeren Strecke konnten mangels Verfügbarkeit noch nicht durchgeführt werden. Zukünftige Testfahrten könnten unter Umständen auf der Teststrecke der TU Braunschweig durchgeführt werden, die ganzjährig aufgebaut ist.\\
Im Performance-Vergleich auf Testfahrten schneidet das CNN gut ab. Die Teststrecke kann nicht nur fast fehlerfrei durchfahren werden, im direkten Vergleich mit dem Steuerungsalgorithmus des Carolo-Cup Teams des WS18/19 liegt die hier entwickelte neuronale Steuerung vorn. Ein zusätzlicher Vergleich mit dem Ursprungsnetz DroNet zeigt, das das Fine Tuning die Lenkwinkelbestimmung entscheidend verbessert hat. Einschränkend muss hier angemerkt werden, dass das Fine Tuning eine Anpassung auf einen eingeschränkten, störungsarmen Testbereich gemacht hat.\\
Zusätzlich wird eine grafische Analyse vorgenommen. Die Salient-Objects, also die für die Bestimmung des Lenkwinkels maßgeblichen Teile des Bilds wurden hervorgehoben. So ließ sich zeigen, dass bei dem CNN was einem Fine-Tuning unterzogen wurde, besonders die Fahrbahnmarkierungen ausschlaggebend waren, was der intuitiven Erwartung entspricht. Zusätzlich wurde das Verhalten der entworfenen Steuerung in neuen, nicht trainierten Szenarien betrachtet. In diesen Fehlerszenarien wurde getestet, ob das Fahrzeug die Fahrbahn findet, wenn es entweder nur teilweise oder gar nicht auf der Fahrbahn startet. Es konnte gezeigt werden, dass die Steuerung in nahezu allen Situationen auf die Fahrbahn navigiert und die Fahrspur sauber findet.\\
Einschränkend muss erwähnt werden, dass sich einige Fahrsituationen als problematisch erwiesen. Dazu gehören Kreuzungen und Situationen, in den denen durch den Kamerawinkel Teile der Fahrmarkierung nicht sichtbar sind. Außerdem ist die Framerate der Bildverarbeitung durch das Netz ein begrenzender Faktor, bei Geschwindigkeiten über \SI{1.2}{\meter/\second} nimmt die Anzahl der Fahrfehler stark zu.\\
Ergänzend kommt hinzu, dass das Generieren eines qualitativ hochwertigen Traningsdatensatzes durch die geringe Länge der Teststrecke erschwert wurde. Die aufgenommenen Bilder enthielten saubere Fahrsituationen, die zwar die Fahrt auf einer Modellstrecke ermöglichen konnten, auf einer Strecke mit fehlender Fahrbahnmarkierung oder ungewöhnlicher Fahrbahnführung zu Problemen führen könnten. Eine Untersuchung hierzu könnte auf einer größeren Teststrecke stattfinden, in der auch Straßensituationen wie zum Beispiel Kreuzungen verschieden modelliert werden können.\\ 
Im Rückblick muss auch die Idee im Grundsatz einer kritischen Betrachtung unterzogen werden. Zwar navigiert das Fahrzeug sicher durch den Testkurs, die urspüngliche Idee entstand jedoch in Anlehnung an die Carolo-Cup Projektarbeit. Das sichere Navigieren ist nur die Grundlage in diesem Wettbwewerb, darauf bauen eine Reihe weiterer Aufgaben, wie zum Beispiel Ein- und Ausparken, Hindernissen ausweichen und Zebrastreifen erkennen auf. Diese Aufgabenstellungen können durch die hier entwickelte neuronale Steuerung nicht abgedeckt. Die Lösung all dieser Aufgaben mit einem einzigen neuronalen Netz wäre aufgrund der Komplexität der einzelnen Aufgabenbereiche kaum möglich. Zudem bieten sich für einige dieser Aufgaben effizientere Lösungen als neuronale Netze.\\
Dennoch könnte für aufbauende Versuche interessant sein, Hindernisse oder andere Verkehsteilnehmer in das Traning mit einzubauen. Hier stellt sich dann allerdings die Frage, wie diese Situationen in der Modellgröße modelliert werden können und wie die benötigten Daten einer sauberen Fahrt durch diese Szenarien generiert werden. Außerdem wäre eine Anpassung im Netzwerkoutput ein spannender Ansatz. Ein zusätzlicher Output für die gefahrene Geschwindigkeit könnte bedeutenden Einfluss auf das Fahrverhalten haben. Ein Abbremsen in engen Kurven und Beschleunigen auf Geraden könnte erwartet werden.\\
Zusammenfassend kann also festgestellt werden, dass das Steuern durch eine markierte Fahrbahn mit einem neuronalen Netz möglich ist, dazu aber Einschränkungen bei der Aufgabenstellung gemacht werden müssen. Das Fahren auf einer Fahrbahn, auch im Modellbereich, ist komplex und in seiner Gesamtheit (noch) nicht durch neuronale Netze abzudecken, dazu ist die Menge der Erscheinungsbilder einer Fahrbahn oder Straße zu groß. Die Einflussfaktoren auf das Verhalten eines Fahrzeuges sind zahlreich, insbesondere im realen Straßenverkehr. Das Ableiten des Lenkverhaltens nur aus der Fahrbahnmarkierung funktioniert für ein eingeschränktes Testszenario alledings gut.\\
Die Entwicklung autonomer Fahrzeuge ist weltweit in vollem Gang, wie sich die Rolle neuronaler Netze in diesem Zusammenhang entwickelt kann mit Spannung beobachtet werden. Vielleicht kann mit zukünftigen Modellen eine rein neuronale Steuerung gelingen.

