% first example chapter
% @author Jan Robert Rösler 
%
\chapter{Neuronale Navigation mit Bilddaten}

\section{Ansätze}
\note{Hier werden Paper vorgestellt, die Geschichte der Navigation auf Bilddaten und wie das fumktiojneiren kann --> Ansätze}

Versuche durch neuronale Verarbeitung von reinen Bilddaten in einem Szenario zu navigieren, gabe es bereits 1989 \cite{pomerleau1989alvinn}.
Das Netzwerk ALVINN (Autonomous Land Vehicle In a Neural Network) sollte das NAVLAB steuern, ein Testfahrzeug für Autonome Navigation der Carnegie Mellon University.
In \ref{img:ALVINN} lässt sich die Architektur nachvollziehen. 


\begin{figure}
	\centering
	\includegraphics[scale=0.8]{figures/Architecture-ALVINN.png}
	\caption{ALVINN Architektur}
	%Quelle: \protect\citeI{Architecture-ALVINN}
	%\caption*{Quelle:}
	\label{img:ALVINN}
\end{figure}

 Präsentation ALVINN, dann gegenüberstellung mit modernem Netzwerk a la NVIDIA.

Kurzer Blick auf  Self driving car steering angle4 prediction und berkeley (large scale video sets) (vielleicht auch später)







